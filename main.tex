\documentclass[12pt,letterpaper]{article}
\usepackage{preamble}
\usepackage{nicefrac}
\usepackage[T1]{fontenc}
%%%%%%%%%%%%%%%%%%%%%%%%%%%%%%%%%%%%%%%%%%
%%%% Edit These for yourself
%%%%%%%%%%%%%%%%%%%%%%%%%%%%%%%%%%%%%%%%%%
\newcommand\course{Principles of Statistical Data Analysis}
\newcommand\hwnumber{3}
\newcommand\userID{Group 15}
\newcommand{\Expl}[1]{&& \text{#1}}
\begin{document}
\begin{table}[h!]
\begin{tabular}{ lc|lc } 
\multicolumn{4}{c}{Effort distribution}\\
 \hline
 Jan \textsc{Alexander} & 25\% & Brecht \textsc{Dewilde} & 25\%\\ 
Arthur \textsc{Leloup} & 25\% & Brecht \textsc{Seifi} & 25\% \\ 
 \hline
\end{tabular}
\end{table}



\section{R function: \texttt{median.test(x,y)}}
The function \texttt{median.test.R} is given on the following pages.
The included R comments can guide the reader through the code.

\lstinputlisting[caption={Function median.test}]{R_scripts/MedianTest.R}
        

\section{Comparison to other tests}

The constructed median test is compared to two other tests:
\begin{enumerate}
	\item the Wilcoxon-Mann-Whitney test
	\item the permutation t-test
\end{enumerate}

\subsection{Power}



\subsection{Type I error}

\newpage  
\section*{Addendum: full code}
\lstinputlisting[caption={Code used to obtain the results discussed in this report}]{HW3_G15.R}

\end{document}
